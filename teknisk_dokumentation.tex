\documentclass[a4paper,titlepage,12pt]{article}

%% Language and font encodings
\usepackage[swedish]{babel}
\usepackage[utf8]{inputenc}
\usepackage{textcomp}
\usepackage{amsmath}
\usepackage{graphicx}
\usepackage[colorinlistoftodos]{todonotes}
\usepackage{afterpage}
\usepackage[colorlinks=true, urlcolor=blue, linkcolor=black, pdfborder={0 0 0}]{hyperref}
\usepackage{longtable}
\usepackage[yyyymmdd]{datetime}
\usepackage[bottom]{footmisc}
\usepackage{titling}
\usepackage{pbox}
\usepackage{booktabs}
\usepackage{color, colortbl}
\definecolor{Gray}{gray}{0.9}
\usepackage{changepage, titlesec}

%Set page size
\usepackage{geometry}
\geometry{margin=3cm}
\usepackage{parskip} 
\setcounter{secnumdepth}{5}


\renewcommand{\dateseparator}{-}
\addto\captionsswedish{% Replace "swedish" with the language you use (if using babel)
  \renewcommand{\contentsname}%
    {Innehållsförteckning}%
}

%%%%%%%%%%%%%%%%%%%%%%%%%%%%%%%
% Header and footer
%%%%%%%%%%%%%%%%%%%%%%%%%%%%%%%

\usepackage{fancyhdr}
\pagestyle{fancy}

\lhead{\includegraphics[scale=0.3]{images/logga.png}}
\chead{System för 3D-kopiering}
\rhead{\today}
\setlength\headheight{26pt} 


\lfoot{TDDD96 --- PUM}
\rfoot{Grupp 9}

\posttitle{\end{center}}

\begin{document}
\begin{titlepage}

% Title text
\topskip0pt
\vspace*{\fill}
\huge
\textbf{Teknisk dokumentation} \\
\Large
System för 3D-kopiering \\
\noindent\rule{17cm}{0.4pt}
\bigskip

\small
\textbf{Hampus Dunström \\
Olof Holmberg \\
Gustav Jannering \\
Michael Karlsson \\
Martin Lundberg \\
Hannes Tuhkala \\
Fredrik Wallström}
\bigskip
\bigskip

Handledare: Sam Le \\
Examinator: Kristian Sandahl

\date{\today}
\vspace*{\fill}

% Image 
\vspace*{\fill}
\begin{minipage}[b]{0.7\textwidth}
	\includegraphics[width=8cm]{images/liu-logga.png}
\end{minipage}
\begin{minipage}[b]{0.4\textwidth}
	\normalsize
	\textit{Linköpings universitet} \\
	\textit{SE-581 83 Linköping, Sverige}\\
	\textit{013-28 10 00, www.liu.se}
\end{minipage}

\end{titlepage}
\newpage
\begin{center}
\pagenumbering{gobble}


%%%%%%%%%%%%%%%%%%%%%%%%%%%%%%%%%%%%%%%%%%%%%%%%%%%%%%%%%%%%%%%%%%%%%
%						Medlemmar
%%%%%%%%%%%%%%%%%%%%%%%%%%%%%%%%%%%%%%%%%%%%%%%%%%%%%%%%%%%%%%%%%%%%%
\section*{\centering Projektidentitet}

System för 3D-kopiering, VT 2017, PUM-9 \\ 
Linköpings Tekniska Högskola, IDA
  
\bigskip

\scalebox{0.8}{
\begin{tabular}{|l|l|l|l|}
	\hline
    \rowcolor{Gray}
	\textbf{Namn} & \textbf{Ansvar} & \textbf{Telefon} & \textbf{E-post} \\
	\hline
    Hampus Dunström & Utvecklingsledare & 0763279292 & hamdu013@student.liu.se\\
    \hline
    Olof Holmberg & Testledare & 0725227007 & oloho254@student.liu.se\\
    \hline
    Gustav Jannering & Analysansvarig & 0706567093 & gusja113@student.liu.se\\
    \hline
    Michael Karlsson & Teamledare & 0735807120 & micka199@student.liu.se\\
    \hline
    Martin Lundberg & Arkitekt & 0762436905 & marlu819@student.liu.se\\
    \hline
    Hannes Tuhkala & Dokumentansvarig \& konfigurationsansvarig & 0760175790 & hantu447@student.liu.se\\
    \hline
    Fredrik Wallström & Kvalitetssamordnare & 0705826972 & frewa814@student.liu.se\\
    \hline
\end{tabular}}

\bigskip
\textbf{Beställare}: Maria Magnusson, maria.magnusson@liu.se 
\\Linköpings universitet ISY

\newpage
\tableofcontents
\newpage

%%%%%%%%%%%%%%%%%%%%%%%%%%%%%%%%%%%%%%%%%%%%%%%%%%%%%%%%%%%%%%%%%%%%%
%					Dokumenthistorik
%%%%%%%%%%%%%%%%%%%%%%%%%%%%%%%%%%%%%%%%%%%%%%%%%%%%%%%%%%%%%%%%%%%%%
  \section*{\centering Dokumenthistorik}
  \renewcommand*{\arraystretch}{1.4}
 \centering
\scalebox{1.1}{
\begin{tabular}{|l|l|l|l|}
	\hline
    	\rowcolor{Gray}
	\textbf{Version} & \textbf{Datum} & \textbf{Utförda förändringar} & \textbf{Utförda av} \\
	\hline
    	1.0 & 2017-06-01 & Första version & Gruppen \\
    	\hline
\end{tabular}}
\end{center}
\newpage

%%%%%%%%%%%%%%%%%%%%%%%%%%%%%%%%%%%%%%%%%%%%%%%%%%%%%%%%%%%%%%%%%%%%%
%					Introduction
%%%%%%%%%%%%%%%%%%%%%%%%%%%%%%%%%%%%%%%%%%%%%%%%%%%%%%%%%%%%%%%%%%%%
\pagenumbering{arabic}
\section{Inledning}
	Detta dokument är en teknisk dokumentation över systemet för 3D-kopieringsprojektet på Linköpings universitet. Projektet pågick under vårterminen 2017 och utfördes av sju stycken studenter ifrån universitetet. Projektet är en del i kursen TDDD96 - Kandidatprojekt i programvaruutveckling som ges vid Linköpings universitet. Den tekniska dokumentationen beskriver hur produkten är utvecklad och hur den fungerar.
	\subsection{Parter}
		Under projektets gång fanns följande parter aktiva:
		
		\begin{itemize}
			\item Beställare \& Kund: Maria Magnusson, Linköpings universitet ISY
			\item Examinator: Kristian Sandahl
			\item Handledare: Sam Le
			\item Projektgrupp: 3DCopy
		\end{itemize}
		
	\subsection{Definitioner}
		Nedan följer definitioner och ordval som används i denna användarmanual.
		
		\begin{itemize}
			\item ISY - Institutionen för systemteknik
			\item ICP - Iterative closest point (iterativ närmast punkt)
			\item PCL - Point cloud library
		\end{itemize}
    
\newpage  

\section{Systemöversikt av 3DCopy}
Systemet består i huvuddel av två stora komponenter: en registreringskomponent och en meshningskomponent. Registreringskomponenten har som syfte att registrera punktmoln så att de bildar en komplett eller fullständig modell av ett punktmoln. Meshningskomponenten har som syfte att generera en 3D-mesh utifrån det kompletta punktmoln som registrerats i registreringskomponenten eller utifrån ett annat komplett punktmoln. Systemet innehåller också ett filter vars syfte är att filtrera olika punktmoln vid behov. Detta för att ta bort onödiga/redundanta punkter. Det finns också ett CLI och ett GUI för att kunna använda själva programmet. 

\newpage

%%%%%%%%%%%%%%%%%%%%%%%%%%%%%%%%%%%%%%%%%%%%%%%%%%%%%%%%%%%%%%%%%%%%%
%					Systemet
%%%%%%%%%%%%%%%%%%%%%%%%%%%%%%%%%%%%%%%%%%%%%%%%%%%%%%%%%%%%%%%%%%%%%
\section{Systemet/3Dcopy} % Välj namn
	Nedan gås systemet igenom mer ingående.
	
	\subsection{Registreringskomponent}
		Registreringskomponenten är implementerad som en klass i C++. Syftet med registreringen är att generera ett komplett punktmoln utifrån flera icke kompletta punktmoln. Registreringsklassen tar in en vektor med flera icke kompletta punktmoln. Klassen går sedan igenom alla punktmoln i vektorn och registrerar dessa parvis till ett komplett punktmoln. För att uträtta detta används algoritmen ICP ifrån biblioteket PCL. Algoritmens metod för att registrera punktmolnen går ut på att minimera avståndet mellan punkterna i molnen. När sedan ett komplett punktmoln har genererats returneras detta. För registrering finns följande parametrar som kan ändras:
		
		\begin{itemize}
			\item Max Iterations - anger hur många registreringsiterationer som algoritmen får utföra innan den tvingas avsluta.
			
			\item Transformation Epsilon - avgör hur stor förflyttning algoritmen får göra på målmolnet inom en iteration.
			
			\item Max Correspondence Distance - anger hur stort avståndet från en punkt i ursprungsmolnet till motsvarande punkt i målmolnet får vara. Är avståndet större än detta kommer algoritmen ignorera punkten i målmolnet vid registreringsförsöket. 
			
			\item RANSAC Outlier Rejection Threshold - anger hur stort avstånd som en punkt tillåts existera från den antagna matematiska modellen för att användas i registreringen. RANSAC, Random Sample Consensus är en metod för att avgöra om en punkt följer det matematiska mönster som algoritmen letar efter.
			
			\item Euclidean Fitness Epsilon - representerar det fel som accepteras. Är skillnaden i felet mellan två iterationer mindre än det här kommer algoritmen avsluta.
			 
		\end{itemize}
	Dessa parametrar kan ändras för att optimera registreringen för specifika punktmoln. 
	
	\subsection{Meshningskomponent}
		Meshningskomponenten är implementerad som en klass i C++. Syftet med meshningen är att generera en 3D-mesh utifrån ett komplett punktmoln. En 3D-mesh är ett 3D-objekt uppbyggt av polygoner. Denna 3D-mesh används för att göra det möjligt att skriva ut objektet i en 3D-skrivare. 
		
		Meshningsklassen tar in ett komplett punkmoln som parameter och tar sedan ut alla punkters normaler i molnet. För att generera en 3D-mesh används Poisson surface reconstruction (PSR) algoritm i PCL. Algoritmen kräver en uppsättning av 3D-punkter med orienterade normaler för att kunna generera en mesh av objektet. Klassen returnerar sedan den färdiga meshen.
		
		Vid meshning av ett objekt kan olika parametrar justeras, beroende på objekt, för att eventuellt förbättra den genererade meshens utseende. De parametrar som kan justeras presenteras nedan.
		
		\begin{itemize}
			\item \textit{depth} - Sätter det maximala djupet i sökträdet som PSR använder sig av, högre värde betyder bättre resultat men blir mer beräkningstungt.
			\item \textit{solver\_divide} - Denna parameter anger vilket djup ett Gauss-Seidel-solver block ska använda för att lösa Laplacian-ekvationen i PSR. Använda denna parameter hjälper att reducera minneskostnaden till en kostnad för rekonstrueringstid.
			\item \textit{samples\_per\_node} - Specificerar det minsta antalet punkter som PSR sätter i en octree-nod då ytan rekonstrueras. Vid brusig punktdata kommer ett högt värde (6-10) jämna ut meshen, men riskerar att tappa detaljer av objektet. Ett lågt värde (1-5) behåller detaljerna men meshen kan bli mer oslipad.
			\item \textit{point\_weight} -  Detta värde specificerar vikten av punkterna i interpoleringen i PSR.
		\end{itemize}
	
		Den genererade meshen kommer med stor sannolikhet inte se ut som man önskat, oftast består den av oönskad massa. För att få meshen att se ut som man vill krävs det manuell handpåläggning av den genererade meshen. Att manuellt behandla en mesh fungerar ungefär som när man redigerar en bild i Photoshop eller en CAD-modell i Autodesk Inventor. Det finns flera olika verktyg för att behandla en mesh, bland annat Autodesk Meshmixer, Meshlab och Blender.
		
\newpage

%%%%%%%%%%%%%%%%%%%%%%%%%%%%%%%%%%%%%%%%%%%%%%%%%%%%%%%%%%%%%%%%%%%%%
%					Användargränssnitt
%%%%%%%%%%%%%%%%%%%%%%%%%%%%%%%%%%%%%%%%%%%%%%%%%%%%%%%%%%%%%%%%%%%%% 
\section{Användargränssnitt}
\subsection{Webbapplikation}
En liten disclaimer, i det här avsnittet kommer en del termer användas utan någon direkt förklaring. Detta är för att det tar alldeles för lång tid och förklara och det finns fantastiska förklaringar, tutorials och dokumentationer på internet. Så är det något du inte förstår, googla.

Eftersom registreringen är rätt så beräkningstung och inte är parallelliserad valde vi att göra om GUI:t till en webbapplikation. Applikationen som blev slutresultatet är mer ett bevis på att konceptet fungerar än en färdig webbapplikation. Den fungerar men saknar mycket av den funktionalitet som man skulle vilja ha, till exempel användarhantering. 

Den applikation som finns är skriven i Python med hjälp av ramverket Django. Är du intresserad av hur de fungerar eller inte vet vad det är och ska arbeta vidare med webbapplikationen så bör du läsa på mer om Django på deras hemsida (https://www.djangoproject.com/). Det finns två stycken Django appar som används \textit{frontend} och \textit{jobs}. Webbapplikationen kräver att du kör Python 3.5 eller högre på grund av den funktionalitet som finns.

\textit{Frontend} hanterar templates, css och views. Allt som behövs för att användaren ska kunna interagera med systemet. Bootstrap som är ett bibliotek för CSS och JavaScript som vi använder för att vår frontend ska se fin ut. Applikationen använder bootstrap från en CDN och om den körs lokalt behöver datorn internetåtkomst för att det ska fungera.

\textit{Jobs} tillhandahåller Django models för registrering och meshningsjobb. Dessa exikveras med genom att applikationen kör subprocesser med instanser av 3DCopy. Webbapplikationen tillhandahåller alltså bara en databas för att spara jobb som ska utföras och när de ska utföras görs detta genom kommandoradsgränssnittet för 3DCopy i en egen process.
 
\subsection{Kommandoradsgränssnittet}
Kommandoradsgränssnittets syfte är att göra det möjligt att använda systemet genom kommandoradsanrop. Kommandoradsgränsnittet är implementerad som en klass i C++ och tar alltså in anrop ifrån kommandoraden och utför sedan den önskade funktionen som angivits. Kommandoradsgränsnittet integrerar registrerings- och meshningsklassen till ett körbart program som gör det möjligt att generera en utskriftsbar 3D-mesh utifrån flera stycken icke kompletta punktmoln.

För att ta in kommandon från terminalen används boosts bibliotek. Boost är ett stort C++ bibliotek och det finns massor av information om det på internet. Boost används även för att hantera inläsning av filer. Detta fungerar väldigt bra man kan använda regex och massa bra saker för att beskriva vilka filer man vill att programmet ska ta in. Det enda problemet är att om man specificerar en mapp så tar inte 3DCopy in filerna i ordning utan de tas in slumpmässigt på grund av boost. Rekommenderat är därför att använda regex eller specificera alla filer som 3DCopy ska använda i ordning. Eftersom ordningen är viktig för registreringen.


%%%%%%%%%%%%%%%%%%%%%%%%%%%%%%%%%%%%%%%%%%%%%%%%%%%%%%%%%%%%%%%%%%%%%
%					Skanning och filtrering
%%%%%%%%%%%%%%%%%%%%%%%%%%%%%%%%%%%%%%%%%%%%%%%%%%%%%%%%%%%%%%%%%%%%% 
\section{Skanning och filtrering}
	För att kunna använda den hårdvara som CVL har tillgång till krävs att de punktmoln som fås av systemet TreeD behandlas för att filtrera bort skräpdata och placera objekten rätt. Detta görs av två separata program som tar skanningar och filtrerar dessa så att de kan användas med 3DCopy: TreeD-Wrapper och filtret.
	
	\subsection{Filter}
		Filtret är uppbyggt för att kunna filtrera punktmoln som är insamlade med TreeD och den tillhörande hårdvaran till det systemet. För att kunna användas i 3DCopy ser filtret till att eventuell skräpdata filtreras bort och ser även till att pinnen som håller i föremålet som skannas tas bort. Filtret flyttar även objektet till origo i koordinatsystemet och ser till att det har rätt rotation. Efter filtreringen är punktmolnet redo att registreras och användas i 3DCopy.
		
		Det första steget som filtret utför är att ta bort alla punkter som ligger utanför ett visst intervall av koordinater. Detta gör att punkter som ligger långt utanför det skannade objektet tas bort. Detta tar alltså bort punkter som kommer från väggen bakom objektet och andra punkter som inte kan vara en del av objektet.
		
		Nästa steg är att ta bort enskilda punkter. Detta görs genom att filtret går igenom varje punkt och räknar hur många grannar den punkten har inom en viss radie. Om punkten har för få grannar betyder det att den ligger långt ifrån andra punkter och inte sitter nära något objekt, alltså tas den bort.
		
		Sista steget är att ta bort pinnen som håller i objektet. Då objektet kan sitta på olika höjd i hårdvaran går det inte att filtrera på koordinater rakt av. Istället delas punktmolnet upp i kluster och sedan räknas en \textit{centroid}, ungefär som en tyngdpunkt, ut för varje kluster. Sedan räknas de kluster vars tyngdpunkt ligger under en viss gräns, som kan bestämmas av användaren, som pinnen och resten räknas som det faktiska objektet. Detta utnyttjas även för att placera objektet exakt i origo. Eftersom pinnen sitter i mitten av objektet räknas den transformation ut som behövs för att placera pinnen exakt i origo. Sedan används denna transformation även på objektet vilket resulterar i att objektet alltid ligger väldigt bra placerat inför registreringen.
		
		Filtret är implementerat som en klass kallad \texttt{Filter} i C++ vilket gör att det kan användas i annan kod genom att inkludera \texttt{Filter.h} och använda den klassen. För att underlätta filtreringen av punktmoln finns även ett kommandoradsgränssnitt som kan köras för att filtrera ett eller fler punktmoln.
		
	\subsection{TreeD-Wrapper}
		För att kunna använda TreeD för att ta ett flertal skanningar och samtidigt filtrera och placera dessa rätt i koordinatsystemet finns en TreeD-Wrapper. Funktionaliteten för denna finns i treed\_wrapper.py och den tillhandahåller ett kommandoradsgränssnitt för att köra ett antal skanningar av objektet från olika vinklar och lutningar. Alla skanningar filtreras också med filtret ovan för att ge punktmoln som kan användas för registrering i 3DCopy.
		
		Då TreeD har visat sig inte vara helt stabilt har TreeD-Wrapper gott om felhantering. Om något fel sker kommer programmet stanna och skriva ut ett felmeddelande på terminalen. Detta gör att användaren kan åtgärda felet och sedan starta om TreeD-Wrapper. Den kommer då fortsätta där den var innan felet uppstod vilket gör det möjligt att ta många skanningar trots att det underliggande systemet kraschar då och då.
		
\newpage


\end{document}

