\documentclass[a4paper,titlepage,12pt]{article}

%% Language and font encodings
\usepackage[swedish]{babel}
\usepackage[utf8]{inputenc}
\usepackage{textcomp}
\usepackage{amsmath}
\usepackage{graphicx}
\usepackage[colorinlistoftodos]{todonotes}
\usepackage{afterpage}
\usepackage[colorlinks=true, urlcolor=blue, linkcolor=black, pdfborder={0 0 0}]{hyperref}
\usepackage{longtable}
\usepackage[yyyymmdd]{datetime}
\usepackage[bottom]{footmisc}
\usepackage{titling}
\usepackage{pbox}
\usepackage{booktabs}
\usepackage{color, colortbl}
\definecolor{Gray}{gray}{0.9}
\usepackage{changepage, titlesec}

%Set page size
\usepackage{geometry}
\geometry{margin=3cm}
\usepackage{parskip} 
\setcounter{secnumdepth}{5}


\renewcommand{\dateseparator}{-}
\addto\captionsswedish{% Replace "swedish" with the language you use (if using babel)
  \renewcommand{\contentsname}%
    {Innehållsförteckning}%
}

%%%%%%%%%%%%%%%%%%%%%%%%%%%%%%%
% Header and footer
%%%%%%%%%%%%%%%%%%%%%%%%%%%%%%%

\usepackage{fancyhdr}
\pagestyle{fancy}

\lhead{\includegraphics[scale=0.3]{images/logga.png}}
\chead{System för 3D-kopiering}
\rhead{\today}
\setlength\headheight{26pt} 


\lfoot{TDDD96 --- PUM}
\rfoot{Grupp 9}

\posttitle{\end{center}}

\begin{document}
\begin{titlepage}

% Title text
\topskip0pt
\vspace*{\fill}
\huge
\textbf{Teknisk dokumentation} \\
\Large
System för 3D-kopiering \\
\noindent\rule{17cm}{0.4pt}
\bigskip

\small
\textbf{Hampus Dunström \\
Olof Holmberg \\
Gustav Jannering \\
Michael Karlsson \\
Martin Lundberg \\
Hannes Tuhkala \\
Fredrik Wallström}
\bigskip
\bigskip

Handledare: Sam Le \\
Examinator: Kristian Sandahl

\date{\today}
\vspace*{\fill}

% Image 
\vspace*{\fill}
\begin{minipage}[b]{0.7\textwidth}
	\includegraphics[width=8cm]{images/liu-logga.png}
\end{minipage}
\begin{minipage}[b]{0.4\textwidth}
	\normalsize
	\textit{Linköpings universitet} \\
	\textit{SE-581 83 Linköping, Sverige}\\
	\textit{013-28 10 00, www.liu.se}
\end{minipage}

\end{titlepage}
\newpage
\begin{center}
\pagenumbering{gobble}


%%%%%%%%%%%%%%%%%%%%%%%%%%%%%%%%%%%%%%%%%%%%%%%%%%%%%%%%%%%%%%%%%%%%%
%						Medlemmar
%%%%%%%%%%%%%%%%%%%%%%%%%%%%%%%%%%%%%%%%%%%%%%%%%%%%%%%%%%%%%%%%%%%%%
\section*{\centering Projektidentitet}

System för 3D-kopiering, VT 2017, PUM-9 \\ 
Linköpings Tekniska Högskola, IDA
  
\bigskip

\scalebox{0.8}{
\begin{tabular}{|l|l|l|l|}
	\hline
    \rowcolor{Gray}
	\textbf{Namn} & \textbf{Ansvar} & \textbf{Telefon} & \textbf{E-post} \\
	\hline
    Hampus Dunström & Utvecklingsledare & 0763279292 & hamdu013@student.liu.se\\
    \hline
    Olof Holmberg & Testledare & 0725227007 & oloho254@student.liu.se\\
    \hline
    Gustav Jannering & Analysansvarig & 0706567093 & gusja113@student.liu.se\\
    \hline
    Michael Karlsson & Teamledare & 0735807120 & micka199@student.liu.se\\
    \hline
    Martin Lundberg & Arkitekt & 0762436905 & marlu819@student.liu.se\\
    \hline
    Hannes Tuhkala & Dokumentansvarig \& konfigurationsansvarig & 0760175790 & hantu447@student.liu.se\\
    \hline
    Fredrik Wallström & Kvalitetssamordnare & 0705826972 & frewa814@student.liu.se\\
    \hline
\end{tabular}}

\bigskip
\textbf{Beställare}: Maria Magnusson, maria.magnusson@liu.se 
\\Linköpings universitet ISY

\newpage
\tableofcontents
\newpage

%%%%%%%%%%%%%%%%%%%%%%%%%%%%%%%%%%%%%%%%%%%%%%%%%%%%%%%%%%%%%%%%%%%%%
%					Dokumenthistorik
%%%%%%%%%%%%%%%%%%%%%%%%%%%%%%%%%%%%%%%%%%%%%%%%%%%%%%%%%%%%%%%%%%%%%
  \section*{\centering Dokumenthistorik}
  \renewcommand*{\arraystretch}{1.4}
 \centering
\scalebox{1.1}{
\begin{tabular}{|l|l|l|l|}
	\hline
    	\rowcolor{Gray}
	\textbf{Version} & \textbf{Datum} & \textbf{Utförda förändringar} & \textbf{Utförda av} \\
	\hline
    	1.0 & 2017-06-01 & Första version & Gruppen \\
    	\hline
\end{tabular}}
\end{center}
\newpage

%%%%%%%%%%%%%%%%%%%%%%%%%%%%%%%%%%%%%%%%%%%%%%%%%%%%%%%%%%%%%%%%%%%%%
%					Introduction
%%%%%%%%%%%%%%%%%%%%%%%%%%%%%%%%%%%%%%%%%%%%%%%%%%%%%%%%%%%%%%%%%%%%
\pagenumbering{arabic}
\section{Inledning}
	Detta dokument är en teknisk dokumentation över systemet för 3D-kopieringsprojektet på Linköpings universitet. Projektet pågick under vårterminen 2017 och utfördes av sju stycken studenter ifrån universitetet. Projektet är en del i kursen TDDD96 - Kandidatprojekt i programvaruutveckling som ges vid Linköpings universitet. Den tekniska dokumentationen beskriver hur produkten är utvecklad och hur den fungerar.
	\subsection{Parter}
		Under projektets gång fanns följande parter aktiva:
		
		\begin{itemize}
			\item Beställare \& Kund: Maria Magnusson, Linköpings universitet ISY
			\item Examinator: Kristian Sandahl
			\item Handledare: Sam Le
			\item Projektgrupp: 3DCopy
		\end{itemize}
		
	\subsection{Definitioner}
		Nedan följer definitioner och ordval som används i denna användarmanual.
		
		\begin{itemize}
			\item ISY - Institutionen för systemteknik
			\item ICP - Iterative closest point (iterativ närmast punkt)
			\item PCL - Point cloud library
		\end{itemize}
    
\newpage  

\section{Systemöversikt av 3DCopy}
Systemet består i huvuddel av två stora komponenter: en registreringskomponent och en meshningskomponent. Registreringskomponenten har som syfte att registrera punktmoln så att de bildar en komplett eller fullständig modell av ett punktmoln. Meshningskomponenten har som syfte att generera en 3D-mesh utifrån det kompletta punktmoln som registrerats i registreringskomponenten eller utifrån ett annat komplett punktmoln. Systemet innehåller också ett filter vars syfte är att filtrera olika punktmoln vid behov. Detta för att ta bort onödiga/redundanta punkter. Det finns också ett CLI och ett GUI för att kunna använda själva programmet. 

%%%%%%%%%%%%%%%%%%%%%%%%%%%%%%%%%%%%%%%%%%%%%%%%%%%%%%%%%%%%%%%%%%%%%
%					Systemet
%%%%%%%%%%%%%%%%%%%%%%%%%%%%%%%%%%%%%%%%%%%%%%%%%%%%%%%%%%%%%%%%%%%%%
\section{Systemet/3Dcopy} % Välj namn
	Nedan gås systemet igenom mer ingående.
	
	\subsection{Registreringskomponent}
		Registreringskomponenten är implementerad som en klass i C++. Syftet med registreringen är att generera ett komplett punktmoln utifrån flera icke kompletta punktmoln. Registreringsklassen tar in en vektor med flera icke kompletta punktmoln. Klassen går sedan igenom alla punktmoln i vektorn och adderar dem till ett komplett punktmoln. För att uträtta detta används algoritmen ICP ifrån biblioteket PCL. Algoritmens metod för att registrera punktmolnen går ut på att minimera avståndet mellan punkterna i molnen. När sedan ett komplett punktmoln har genererats returneras detta.
	
	\subsection{Meshningskomponent}
		Meshningskomponenten är implementerad som en klass i C++. Syftet med meshningen är att generera en 3D-mesh utifrån ett komplett punktmoln. En 3D-mesh är ett 3D-objekt uppbyggt av polygoner. Denna 3D-mesh används för att göra det möjligt att skriva ut objektet i en 3D-skrivare. Meshningsklassen tar in ett komplett punkmoln som parameter och tar sedan ut alla punkters normaler i molnet. För att generera en 3D-mesh används Poisson surface reconstruction algoritm. Algoritmen kräver en uppsättning av 3D-punkter med orienterade normaler för att kunna generera en mesh av objektet. Klassen returnerar sedan den färdiga meshen.
	
	\subsection{Filter}
		% Beskriv mer om filtret typ hur det filtrerar (typ vilka gränser osv)
		Filtret är uppbyggt för att kunna filtrera punktmoln som är insamlade med TreeD och den tillhörande hårdvaran till det systemet. För att kunna användas i 3DCopy ser filtret till att eventuell skräpdata filtreras bort och ser även till att pinnen som håller i föremålet som skannas tas bort. Filtret flyttar även objektet till origo i koordinatsystemet. Efter filtreringen är punktmolnet redo att registreras och användas i 3DCopy.
		
\newpage
 
%%%%%%%%%%%%%%%%%%%%%%%%%%%%%%%%%%%%%%%%%%%%%%%%%%%%%%%%%%%%%%%%%%%%%
%					Användargränssnitt
%%%%%%%%%%%%%%%%%%%%%%%%%%%%%%%%%%%%%%%%%%%%%%%%%%%%%%%%%%%%%%%%%%%%% 
\section{Användargränssnitt}
	\subsection{Serverapplikation/webbapplikation}
		%% Skriv hur det är uppbyggt
	
	\subsection{Kommandoradsgränssnittet}
		Kommandoradsgränssnittets syfte är att göra det möjligt att använda systemet genom kommandoradsanrop. Kommandoradsgränsnittet är implementerad som en klass i C++ och tar alltså in anrop ifrån kommandoraden och utför sedan den önskade funktionen som angivits. Kommandoradsgränsnittet integrerar registrerings- och meshningsklassen till ett körbart program som gör det möjligt att generera en utskriftsbar 3D-mesh utifrån flera stycken icke kompletta punktmoln.
	
\newpage


\end{document}

